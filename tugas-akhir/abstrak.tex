\renewcommand*\abstractname{ABSTRAK}
\addcontentsline{toc}{chapter}{Abstrak}
\begin{abstract}
Pada zaman globalisasi saat ini, pertumbuhan industri dan kewirausahaan berkembang dengan sangat pesat. Akibatnya kebutuhan atau mobilisasi ke tempat yang ada di dunia juga meningkat. Hal ini yang mendorong munculnya suatu perusahaan atau manajemen membangun sebuah konsep kepemilikan pesawat. Konsep kepemilikan pesawat ini juga dikenal dengan Fractional Aircraft Ownership  (FAO).
Fractional Aircraft Ownership (FAO) adalah suatu konsep kepemilikan pesawat secara bersama dengan membagi penggunaan pesawat berdasarkan saham yang dimiliki. FAO menawarkan konsep kepemilikan pesawat dengan membeli saham dari perusahaan FAO tersebut tanpa harus membeli pesawat. Pemilik saham dapat menggunakan pesawat sesuai dengan jumlah saham yang dimilikinya. 
Bagi sebagian besar masyarakat, menggunakan pesawat untuk bepergian ke luar kota atau luar negeri tergolong hal yang mewah. Namun, untuk kalangan menengah ke atas atau orang yang memiliki tingkat mobilisasi tinggi dan hampir menggunakan pesawat tiap harinya, hal ini akan menjadi kebutuhan utama. Konsumen yang menggunakan FAO ini adalah orang yang memiliki jam terbang selama lima puluh jam atau lebih per tahunnya.
Untuk menggunakan FAO ini, konsumen hanya perlu membeli bagian saham (share) dari perusahan FAO tersebut. Share yang dimiliki oleh konsumen akan dinikmati dalam bentuk jam terbang dalam satu tahun. Dengan FAO, pemilik saham dapat memiliki waktu yang lebih fleksibel dan tingkat privasi yang lebih tinggi dalam berpergian.
FAO terdiri dari manajemen FAO dan pemilik saham. Pemilik saham atau owner membeli saham dari perusahaan FAO. Sedangkan manajemen FAO adalah pihak yang mengatur dan menyediakan semua kebutuhan owner. Owner dan manajemen FAO terikat dalam suatu surat perjanjian. Segala ketentuan atau aturan-aturan yang berlaku diatur dalam surat kontrak yang telah disetujui oleh pihak manajemen dan owner. Owner dan manajemen FAO memiliki hak dan kewajiban masing-masing. Hak dan kewajiban ini ditulis dalam surat kontrak yang telah disepakati bersama. 
Manajemen FAO akan melayani sejumlah owner dan masing-masing owner mempunyai share atau bagian saham masing-masing. Selain itu, owner-owner tersebut mempunyai demand atau permintaan. Manajemen tidak bisa memprediksi jumlah owner, share, dan demand karena bersifat stokastik. Sehingga tugas manajemen disini adalah memenuhi semua permintaan owner dengan biaya yang minimum.


\end{abstract}